\documentclass[12pt,a4paper]{article}
\usepackage[utf8]{inputenc}
\usepackage[spanish]{babel}
\usepackage[margin=0.5in, top=0.5in, bottom=0.5in]{geometry}
\usepackage{amsmath}
\usepackage{amsfonts}
\usepackage{amssymb}
\usepackage{hyperref}
\usepackage{graphicx}
\usepackage[shortlabels]{enumitem}
\newcommand{\p}{\phantom{......}}

\title{Bases de datos 2023-1\\
Práctica 2: Análisis de requerimientos}
\begin{document}
\maketitle

\section{Análisis de requerimientos}
``El Gran Vivero'' necesita de una herramienta para almacenar sus datos
mientras espera a que se le desarrolle una base de datos permanente.\\

\begin{itemize}
	\item La herramienta debe de ser poder añadir, consultar, editar y eliminar
		entidades. Las consultas se hacen usando la llave primaria y deben
		regresar la entidad completa.\\

	\item Es necesario que se limpien los datos antes de meterlos a la base y
		se debe asegurar la integridad de estos (en particular de los números).\\

	\item La herramienta debe de ser robusta, ,anejar y comunicar los errores
		de usuario limpiamente y proteger a los datos de los archivos.\\

	\item Los archivos deben ser texto legible en formato \texttt{.csv} .\\

	\item Las entidades que participarán en la base son \textbf{Vivero} (la tienda),
		\textbf{Empleado} (los trabajadores de los viveros) y \textbf{Plantas}.\\

	\item Para las plantas debemos almacenar:\\
		\begin{itemize}
			\item Precio.\\
			\item Cantidad que cabe en un invernadero
				(Aquí asumimos que refiere a cuantas pueden crecer en un invernadero,
				no cuantas hay almacenadas en un vivero).\\
			\item Nombre común (Actúa como identificador).\\
			\item Género de la planta (pensando en nombre científico).\\
			\item Si es africana o un cactus.\\
			\item Los cuidados que debe de recibir.\\
			\item El tipo de sustrato (alcalino, arcilla húmeda, etc).
				Asumimos que puede haber cientos de ellos así que los
				almacenaremos en una cadena.\\
			\item Si la planta es de sol, de sombra o resolana.\\
			\item Fecha de germinación (Suponemos que todas se plantan al mismo tiempo
				ya que se usan invernaderos para crecerlas).\\
			\item Horas entre riegos.\\
			\item Volumen de agua en cada riego.\\
		\end{itemize}
		
	\item Para los empleados:\\
		\begin{itemize}
			\item Identificador.\\
			\item Nombre completo.\\
			\item Dirección de residencia.\\
			\item Correos electrónicos (Puede haber más de uno).\\
			\item Teléfonos (Puede tener más de uno).\\
			\item Fecha de nacimiento.\\
			\item Salario.\\
			\item Rol en el vivero (Gerente, Cuidador, Presentador o Cajero).\\
			\item Nombre del Vivero en el que trabaja.\\
		\end{itemize}

	\item Para los viveros:\\
		\begin{itemize}
			\item Nombre del vivero (Identificador).\\
			\item Dirección del vivero.\\
			\item Números telefónicos (puede haber más de uno).\\
			\item Fecha de apertura.\\
		\end{itemize}

	\item Un empleado trabaja en un solo vivero y tiene un único rol.
		Esto ya esta reflejado en nuestras descripciones de entidad.\\
\end{itemize}



\section{Preguntas}
\begin{itemize}
	\item \textbf{Menciona 5 diferencias entre almacenar la información
		utilizando un sistema de archivos a almacenarla utilizando una base de datos.}\\
		\begin{enumerate}
			\item 1 Los archivos hacen complicado escribir en partes aleatorias de ellos
				sin eliminar datos. Las bases de datos no.\\

			\item 2 Una base de datos asegura mantener las restricciones de cardinalidad
				de nuestra información y los archivos no.\\

			\item 3 En un sistema de archivos solo podemos almacenar una cantidad finita 
			de información por archivo, mientras que en una base de datos podemos 
			almacenar tanta como podamos crecer nuestra capacidad de hardware de
			almacenamiento. \\
			\item 4 En las bases de datos podemos restringir diferentes permisos de 
			almacenamiento a distintos usuarios, brindándole mayor seguridad a nuestra 
			información en una base de datos que a un sistema de archivos. \\
			\item 5 En las bases de datos, por medio del SMBD, garantizamos la integridad 
			de los datos ya no permitimos múltiples y simultáneas modificaciones a la base, 
			impidiendo de esta forma la fragmentación de la realidad y de la información. \\
		\end{enumerate}


	\item \textbf{Describe cual es mas conveniente utilizar (sistema de archivos o base de datos).}\\

		La conveniencia de una u otra tecnología depende de donde la usamos.
		No es lo mismo un problema con datos altamente estructurados donde
		constantemente queremos escribir, borrar, modificar y leer comparada
		con otra en la que solo pensamos escribir una sola vez.\\

		Como ejemplo podemos considerar el representar un árbol binario, la base de
		datos va a resultar algo incomoda para manipular el árbol mientras
		que un archivo \texttt{.xml} va a ser más cómodo.\\
\end{itemize}

\end{document}
