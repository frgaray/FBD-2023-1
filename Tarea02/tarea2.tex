\documentclass[12pt,a4paper]{article}
\usepackage[utf8]{inputenc}
\usepackage[spanish]{babel}
\usepackage[margin=0.5in, top=0.5in, bottom=0.5in]{geometry}
\usepackage{amsmath}
\usepackage{amsfonts}
\usepackage{amssymb}
\usepackage{hyperref}
\usepackage[shortlabels]{enumitem}
\newcommand{\p}{\phantom{......}}

\title{Bases de datos 2023-1\\
Tarea 2: Modelo Entidad-Relación}
\begin{document}
\maketitle
\begin{enumerate}

	\item Conceptos del Modelo Entidad – Relación
		\begin{enumerate}
			\item ¿Qué es un tipo de relación? Explica las diferencias con respecto a una instancia de relación.\\

				El tipo de relación describe como es la relación y que tipos de entidades participan,
				mientras que la instancia es la asociación entre instancias de entidades particulares.\\

			\item ¿Bajo qué condiciones se puede migrar un atributo de algún tipo de entidad que
				participa en un tipo de relación binaria y convertirse en un atributo del tipo de
				relación? ¿Cuál sería en el efecto?\\

				Es necesario que el atributo describa algo que pertenga a todas las entidades
				que la relación asocia. Cambiarlo fuerza algunas propiedades de la relación,
				por ejemplo si es necesario que una entidad tenga acceso a este entonces
				la relación debe tener participación total.\\

			\item ¿Cuál es el significado de un tipo de relación recursiva?
				Proporciona un par de ejemplos de este tipo de relación.\\

				Una relación recursiva es una donde un tipo de entidad participa
				más de una vez.Como ejemplo podemos considerar la entidad \texttt{Persona}
				y la relación \texttt{Casar} ya que en ella participan dos entidades de
				persona diferentes.Otro ejemplo es el de \texttt{Empresa} y \texttt{Contrata}
				ya que una empresa puede contratar a otras empresas.\\

			\item Responde a las siguientes cuestiones, deberás indicar si son posibles o no,
				justificando tu respuesta. Cuando no sea posible deberás indicar alguna recomendación al respecto:
				\begin{itemize}
					\item ¿Un atributo compuesto puede ser llave?\\

						Sí, la llave solo debe de identificar a la instancia de entidad,
						a veces es necesario más de un valor para hacerlo (teléfono y código de área).

					\item ¿Un atributo multivaluado puede ser llave?\\

						Podría serlo si cada uno de sus valores fuera único respecto
						al resto de entidades ya que conocer uno de los elementos
						en el multivaluado nos dice de que entidad hablamos,
						pero seguramente hay razones
						de implementación para no permitir esto (índices).

						Si necesitáramos hacerlo podríamos definir un valor ``principal''
						que sea la llave y que el resto estén en su propio multivaluado.\\

					\item ¿Un atributo derivado puede ser llave?\\

						Sí, podemos imaginar que del resto de los atributos
						de la entidad calculamos un valor que debe ser único
						para esta. (Por ejemplo obtener una CURP de otros valores)\\

					\item ¿Un atributo multivaluado puede ser compuesto?\\

						Sí, cada valor del multivaluado se compone en sí
						de otros valores. \\

					\item ¿Un atributo multivaluado puede ser derivado?\\

						Sí, si derivamos de un atributo multivaluado probablemente
						vamos a tener varios valores y necesitaremos otro multivaluado
						para representarlos.\\

					\item ¿Qué implicaría la existencia de una entidad cuyos atributos sean todos derivados?\\

						Si una entidad tuviera solo valores derivados, entonces solo puede depender
						del estado del ``sistema'' para calcularlos, por lo que
						la instancia de entidad que reside ahí tendría que ser única.\\
				\end{itemize}

			\item Explica el concepto de categorías (herencia múltiple) en el modelo E-R
				y proporciona dos ejemplos de la vida real en donde se aplique este concepto.\\

				Las categorías existen por que comúnmente se repiten atributos entre
				entidades distintas, estas entidades son particularizaciones de una entidad
				que solo tiene los atributos compartidos ``padre''.\\

				Ejemplos:\\
				Consideremos trenes y sus conductores, los trenes y los
				conductores ambos tienen horarios de trabajo. Los conductores
				al igual que los cocineros del vagón restaurante son empleados.
				Entonces \texttt{Conductor} hereda de \texttt{Horario} y de \texttt{Empleado}.\\

				Podríamos tener una base de datos de diferentes dispositivos electrónicos y eléctricos.
				En ella podría haber una entidad \texttt{NetworkRouter} que sea una categoría
				específica de \texttt{DispositivoElectrónico}, de \texttt{DispositivoTelecomunicación}
				y de \texttt{DispositivoParaEstante}.\\

		\end{enumerate}

	\item 2
	\item 3
	\item Modificar modelo Entidad/Relación de cadena de farmacias.\\
		El diagrama del modelo modificado está en \texttt{Diagramas/Ejercicio4.png}.\\

		Consideraciones:\\
		\begin{enumerate}
			\item Cada paciente debe tener un único médico de atención primaria. Cada médico tiene al menos un paciente
				y queremos conocer la cédula, nombre y especialidad.\\

				Dejamos a paciente como parcial en la relación considerando que
				la base de datos tal vez ya tenía pacientes sin doctor que ahora
				deben de poder estar en este nuevo modelo.\\

			\item El diseño solo modela el hecho de que un paciente toma algunos medicamentos, queremos que ahora el modelo
				permita que un paciente tome ciertos medicamentos recetados por un médico y la fecha de prescripción.\\

				Habíamos considerado usar una relación ternaria pero preferimos evitarla.
				Mejor decidimos añadir una nueva entidad receta que liga al doctor, al paciente y a los medicamentos recetados.\\

				Además el atributo \texttt{Cantidad} que estaba en la relación \texttt{Tomar} ahora esta
				en la relación \texttt{Prescribir} que liga \texttt{Receta} y \texttt{Medicina}.\\

			\item Las compañías farmacéuticas tienen contratos a largo plazo con farmacias. Un compañía farmacéutica establecer
				contratos con varias farmacias, y una farmacia puede contratar varias compañías farmacéuticas. Para cada
				contrato, queremos almacenar la fecha de inicio y la fecha de finalización.\\

				Estamos suponiendo que las farmacias necesitan tener contratos para tener productos para vender y
				que las farmacéuticas los necesitan para poder comercializar sus medicinas.\\

		\end{enumerate}
\end{enumerate}
\end{document}
