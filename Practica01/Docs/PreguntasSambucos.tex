\documentclass[12pt,a4paper]{article}
\usepackage[utf8]{inputenc}
\usepackage[spanish]{babel}
\usepackage[margin=0.5in, top=0.5in, bottom=0.5in]{geometry}
\usepackage{amsmath}
\usepackage{amsfonts}
\usepackage{amssymb}
\usepackage{hyperref}
\usepackage{graphicx}
\usepackage[shortlabels]{enumitem}
\newcommand{\p}{\phantom{......}}

\title{Bases de datos 2023-1\\
Práctica 1: Preguntas}
\begin{document}
\maketitle

\begin{enumerate}
    \item ¿Qué otros SMBD existen actualmente en el mercado?\\
        \begin{itemize}
            \item MySQL + MariaDB
            \item Oracle
            \item Microsoft SQL Server
            \item SQLite
        \end{itemize}

    \item ¿Cuáles son las principales diferencias con PostgreSQL?\\
        \item Comportamiento NoSQL, como parejas llave-valor y compatibilidad con
        JSON; licencia de código abierto (MIT); sintaxis poco compleja, cercano al
        estándar SQL; tipos de datos avanzados tal como arreglos, hora-tiempo, XML
        entre otros.

    \item ¿Por qué una empresa debería escoger una base de datos open source?\\
        Una opción de código abierto trae dos grandes ventajas:
        Primero, si la desarrolladora deja de soportar la base de datos la empresa
        puede mantenerla por sí misma.\\
        También en caso de tener problemas o necesidades especiales se puede
        modificar la base de datos a las necesidades de la empresa.\\

    \item ¿Cuáles son las ventajas, para un DBA el trabajar con un SMBD, open source?\\

        La mayor ventaja es que existe una comunidad en linea mucho mas grande y al haber
        considerablemente más usuarios y gente con acceso al código, los bug fixes llegan bastante rápido
        y en general son muy estables. Además por las mismas razones es fácil encontrar apoyo en línea.

    \item Describir a detalle qué es y para que sirve Docker y dar al menos 2 ejemplos\\
        de cómo podemos utilizar esta herramienta.

        Docker es como una máquina virtual pero comparte su kernel con un sistema operativo
        anfitrión Linux, esto nos deja tener la comodidad de tener sistemas separados
        corriendo diferentes programas como servidores http o dns sin tener que emular
        completamente un sistema operativo.\\

        En Docker es muy fácil describir configuraciones de sistemas y compartirlos
        con otros usuarios, además se puede usar software adicional para tener muchas
        instancias del mismo programa corriendo simultáneamente.\\

        Primer ejemplo: Usando coordinación de contenedores se puede tener un servicio
        web separado en muchos contenedores cada uno con su propia configuración.
        Conforme aumenten o disminuyen las peticiones al servidor podemos ejecutar
        más de un contenedor del mismo programa para soportar la demanda.
        (Necesitando de un balanceador de carga, caches ...)\\

        Segundo ejemplo: Se puede usar para distribuir aplicaciones o servicios
        web de uso personal, los usuarios sólo necesitan correr el contenedor
        para tener acceso al programa, por ejemplo \texttt{pi-hole} o \texttt{gitea}.\\

    \item ¿Qué son las bases de datos NoSQL? Menciona 3 ventajas y desventajas contra
        las bases relacionales.\\

        NoSQL (Not Only SQL) es un término que en general se utiliza para hablar
        sobre bases de datos no relacionales, que aparecieron en la primera década
        de este siglo para satisfacer las demandas que presentaba "la era de la
        información", y resolver los problemas que aparecieron junto a la gran cantidad
        de datos de esta misma.\\
        
        \textbf{Ventajas:}\\
        \begin{enumerate}
            \item Rendimiento: se evita el acceder a varias tablas, y hacer joins entre ellas, con lo que en algunos casos pueden tener mejor rendimiento.
            \item Flexibilidad: no tienen la misma rigidez de una base de datos relacional, por lo que pueden adaptarse a cambios en los datos sin tener que cambiar la estructura de la base de datos.
            \item Escalabilidad: mientras que en las bases de datos relacionales para aumentar la capacidad suele ser necesario migrar a un servidor que cumpla nuestras nuevas necesidades, muchas bases de datos NoSQL permiten agregar capacidad sin necesidad de cambiar de servidor.
        \end{enumerate}
        
        \textbf{Desventajas:}\\
        \begin{enumerate}
            \item Falta de estandarización: usamos el término NoSQL para referirnos a un conjunto muy variado de soluciones, por lo que podemos encontrarnos con productos muy distintos unos de otros que caen en esta categoría (ej. almacenamiento llave - valor [DynamoDB] y bases de datos de archivos [MongoDB]).
            \item Falta de madurez: ya que son productos relativamente nuevos y usados solamente en nichos de mercado específicos, son productos para los cuales es más complicado encontrar documentación, expertos y ayuda en general cuando aparece un problema.
            \item Consistencia:muchas soluciones NoSQL sacrifican la consistencia en el intento de conseguir velocidad, escalabilidad y disponibilidad.
        \end{enumerate}

\end{enumerate}


\end{document}
