\documentclass[12pt,a4paper]{article}
\usepackage[utf8]{inputenc}
\usepackage[spanish]{babel}
\usepackage[margin=0.5in, top=0.5in, bottom=0.5in]{geometry}
\usepackage{amsmath}
\usepackage{amsfonts}
\usepackage{amssymb}
\usepackage[shortlabels]{enumitem}
\newcommand{\p}{\phantom{......}}

\title{Bases de datos 2023-1\\
Tarea 1: Conceptos básicos}
\begin{document}
\maketitle

\begin{enumerate}
	\item \textbf{Conceptos generales}
		\begin{enumerate}
			\item Describe las principales características del enfoque de bases de datos y contrástalo con el enfoque
				basado en archivos. ¿En qué casos no tendría sentido utilizar una base de datos?

			\item ¿Qué ventajas y desventajas encuentras al trabajar con una base de datos?
			
                        	Las ventajas son la consulta, actualización y almacenamiento de datos, ya que de hecho ese es su prinicipal propósito y 			        pueden almacenar cantidades muy grandes de información y permiten la rápida consulta y actualización de esta, por otro                                   lado también tenemos la validación de dichos datos, la seguridad y el tener todos los datos en un mismo sistema en vez                                   de varios sistemas que no se puedan sincronizar, validar o cualquier problema de ese estilo.\\ \\
    
    			 	Entre las principales desventajas se encuentran\\
      				\begin{itemize}
     				\item La falta de personal \textbf{capacitado} para manipular dichas bases de datos.
   			        \item El costo, ya que aparte del costo del desarrollo de estas también está el costo del servidor en el que se va a 	                                 alojar la base de datos.
    				\item El peso de estas puede llegar a ser muy grande ya que se busca tener los datos centralizados, y puede llegar a                                     alentarse si ésta crece mucho (más de lo esperado inicialmente).
   			        \item require personal de mantenimiento.
                                \item Hay que actualizar la base de datos según las necesidades lo requieran.
                                \end{itemize}				
				
			\item ¿Qué es la independencia de datos? ¿Cuál tipo de independencia de datos es más difícil de lograr?
				Justifica tu respuesta.

			\item Explica la diferencia entre los esquemas externo, interno y conceptual. ¿Cómo se relacionan estas
				diferentes capas de esquemas con los conceptos de independencia de datos lógica y física?

				El esquema externo tiene que ver con las vistas que cada usuario tiene de los datos.
				
				El esquema interno tiene que ver con la descripcion fisica de los datos.
				
				El esquema conceptual se encuentra, de cierta manera, "entre estos 2", pues es una desscripcion logica 
				(como el esquema externo), unica y global (como el esquema interno) de nuestros datos.

			\item Investiga qué es la redundancia de datos. ¿Cuál sería la diferencia entre redundancia de datos
				controlada y no controlada?\\

				La redundancia de datos se refiere a que los datos se encuentran en más de un lugar de
				la memoria que usamos (en el caso de bases de datos, tenemos datos que se repiten en la base).
				Esta redundancia puede ser útil o contraproducente dependiendo de la situación.

				La redundancia controlada es cuando repetimos los datos intencionalmente, lo solemos
				hacer para protegernos de perdidas o fallos en nuestros sistemas o para amplificar las
				capacidades de nuestras aplicaciones.
				Las copias de seguridad \textit{offsite} por ejemplo nos protegen de perdidas de datos
				causadas por accidentes o catástrofes.
				Las caches aprovechan la redundancia para acelerar el tiempo de acceso a recursos,
				por ejemplo una cache en un CDN nos deja servir archivos a muchos clientes sin
				tener que remitirnos al servidor original para satisfacer la petición
				(particularmente útil cuando los datos son estáticos, como una biblioteca de uso general).

				La redundancia no controlada es cuando sin nuestra intención se repiten los datos,
				no podemos aprovecharnos de esto y terminan alentando o obstruyendo nuestras necesidades.

			\item Investiga cuáles son las responsabilidades de un DBA. Si asumimos que el DBA nunca está interesado
				en ejecutar sus propias consultas, ¿necesita entender y/o conocer el modelo de datos lógico de la
				base de datos? Justifica tu respuesta.

			\item Entrevista a algún usuario de sistemas de bases de datos, ¿qué características de SMBD encuentran
				más útiles y por qué? ¿qué instalación(es) de SMBD encuentran más/menos complicada y por qué?
				¿cuáles perciben estos usuarios que son las ventajas y desventajas de un SMBD?
				
			      Entrevistado: Fernando López Balcazar.	
				
			      Algunas de las caracteristicas mas importantes son:
                              
			      - Los drivers:
                              Que los drivers que tengan los lenguajes de programacion con el SMBD y qué tan bien funcionan estos ya que al momento de          		       desarrollar un sistema es importante decidir cuál es el lenguaje que más beneficios o facilidades aporta. Tambien es 				      importante que tengan facilidad al establecer conexiones.

			      - Sistemas de tipos:	
			      Es importante conocer la forma en que el SMBD va a guardar los atributos ya que es importante el uso de memoria.

			      - Soporte y comunidad:
			      Es importante conocer el soporte que el SMBD que estemos utilizando así como qué tan actualizado está.

		              MariaDB ha sido el SMBD que mas me ha costado trabajo instalar debido a su incompatibilidad con MySQL en la distribucion 				      Linux Fedora pues son prácticamente el mismo sistema y trabajan muchas configuraciones iguales. El mas sencillo ha sido 				      PostgreSQL pues fue de manera bastante intuitiva y mi sistema operativo no presento incomptabilidad.

			      Las principales ventajasd e utilizar un SMBD es la versatilidad, seguridad y si estamos en un proyecto lo suficientemente 			      robusto entonces el uso de un SMBD es indispensable.
	
				
			\item Supón que deseas crear un sitio de videos similar a TikTok. Considera cada uno de los puntos
				enumerados en el documento “Purpose of Database Systems”, como desventajas de administrar los
				datos en un sistema de procesamiento de archivos. Discute la relevancia de cada uno de los puntos
				indicados con respecto al almacenamiento de datos de los videos: el usuario que lo subió, la fecha de
				carga, las etiquetas, qué usuarios comentaron, cantidad de “Me gusta”, entre otros.

			\item Investiga por qué surgieron los sistemas NoSQL en la década de 2000 y compara a través de una tabla
				sus características vs. los sistemas de bases de datos tradicionales.

				La llegada del "big data" y el requerimiento de procesar cantidades gigantescas de datos lleva 
				a compañias como amazon (SimpleDB, DynamoDB) y google (Bigtable, MapReduce) a desarollar sistemas 
				manejadores de bases de datos que sean mas rapidos, altamente paralelizables y flexibles. 

			\item Asumiendo que una base de datos es un lugar donde se almacenan datos de forma sistemática y que
				la información se obtiene al consultar los datos entonces, un diccionario puede considerarse como una
				base de datos. Imagina que vas a buscar el significado de la palabra Luminiscencia, indica cómo
				efectuarías la búsqueda y los problemas que enfrentarías con: un diccionario con palabras
				desordenadas, un diccionario con palabras ordenadas (sin índice) y un diccionario con palabras
				ordenadas (con índice).

				Para el diccionario con palabras desordenadas no tenemos de otra más que ir revisando una por una
				las palabras hasta encontrar \texttt{Luminiscencia} pues no tiene estructura que podamos aprovechar.
				Esto puede ser particularmente malo si \texttt{Luminiscencia} resulta ser la última palabra del diccionario.

				Si tenemos las palabras ordenadas podemos usar un algoritmo de búsqueda binaria:
				Abrimos el diccionario a la mitad y vemos si la palabra es mayor o menor en orden
				lexicográfico a la primer palabra de la página, si es mayor hacemos recursión con el lado izquierdo
				del diccionario si es menor con la derecha.
				Al final debemos de terminar entre dos páginas y de ahí seguimos la búsqueda binaria ahora
				sobre la cantidad de palabras.

				Finalmente con el índice solo tenemos que buscar la letra que buscamos y saltar directamente a
				ese capítulo, de ahí hacemos búsqueda binaria sobre las todas las palabras que
				empiezan con la letra \texttt{L} para encontrar \texttt{Luminiscencia}.

		\end{enumerate}

	\item \textbf{Lectura}
		\begin{enumerate}
			\item Leer el resumen ejecutivo ¿Por qué son tan importantes los datos? y realizar un resumen del documento
				(páginas 1 a 16), destacando los puntos que a su consideración sean los más relevantes (no más de
				una cuartilla).

			\item Realizar un ensayo donde expresen sus comentarios (cada integrante del equipo deberá indicar este
				punto de forma individual en el documento que redacten) sobre la lectura, considerando los siguientes puntos :...

				\textbf{Angel:}\\
				Considero que el texto leído es un intento mal realizado de
				persuasión acerca de el valor de los datos.
				No estoy en desacuerdo con que los datos son verdaderamente
				valiosos, pase meses en la lista de espera de DALLE, sé
				de que son capaces los datos masivos, pero siento que
				en la lectura no se toman en serio los nuevos retos
				que nos han traído los datos.

				En pocas palabras, "Por que son importantes los datos"
				es un texto hablando de los beneficios que nos han traído los
				datos y como el pequeño rol que jugaban en el pasado
				se esta volviendo una verdadera oportunidad de progreso,
				se tocan las áreas de recopilación de datos, internet de las cosas,
				algoritmos, análisis de datos y computación en la nube. 

				La lectura se enfoca en las nuevas bondades que han traído
				estos datos a nuestra vida diaria, desde mejores decisiones en los
				negocios hasta beneficios en la medicina.
				A la vez se ignoran los cientos de nuevos problemas que nos han traído
				los datos, algoritmos de recomendación que nos desconectan de nuestros
				seres queridos, inteligencia artificial enfocada al daño de seres
				vivos y la extracción de hasta nuestro último centavo por parte
				de las grandes empresas de comercio.

				La explosión cambrica en los negocios que han causado
				los datos masivos es una espada de doble filo, este cambio
				paradigmático en como nos relacionamos con el mundo no necesariamente
				será positivo a largo plazo, en lugar de la actitud temeraria y
				sobre optimista debemos de ser escépticos y cuidadosos, no
				sería la primera vez que un cambio en nuestra tecnología
				terminará por mordernos la mano.

				\textbf{Ian}\\
				El texto nos informa sobre el gran potencial que tienen los datos, 
				además del gran valor que ya están agregando "actualmente" (2015?) 
				a varios sectores, entre los cuales podemos encontrar el de la medicina, 
				el transporte, la energía, las finanzas, la producción y la agricultura.

				En general el protagonista principal del texto son los datos, pasando
				primero por todos los puntos de su ciclo de vida, su obtención, 
				almacenamiento, análisis y aprovechamiento. para después centrarse 
				en todo lo que estos datos pueden traer a nuestro mundo, desde ahorros 
				económicos, aumentos a la productividad, mejoras en eficiencia y 
				eficacia, e inclusive cosas que hace no mucho tiempo considerábamos 
				imposibles, como vislumbrar un poco hacia lo que nos depara el futuro.

				Aunque me siento optimista con el papel que jugaran los datos a 
				futuro, la realidad es que no podemos seguir pensando que el
				crecimiento de todas las cosas puede ser exponencial indefinidamente, 
				hace casi 2 décadas vimos el final del escalamiento de dennard, 
				de la misma manera estamos cerca del fin de la ley de moore, y tarde
				o temprano también nos toparemos con otros límites físicos, como el 
				principio de Landauer (cota inferior del consumo de energía por
				cálculo), o con el límite teórico máximo de datos que pueden ser
				transmitidos por una fibra óptica.

				No por esto los datos presentan una oportunidad menos interesante,
				son la conclusión lógica de todo el progreso que hemos tenido en
				IT, además de ser una herramienta muy valiosa en todos los aspectos
				de la vida. Los retos que estos presentan, más que ser un motivo
				para desalentarnos, deben ser lo que nos motive a seguir adelante,
				buscando soluciones a los problemas para los que actualmente parece
				no haber solución alguna.
				
				\textbf{Rodrigo}\\	
                      		El autor principalmente nos quiere dar a entender que los datos son 
				y seguirán siendo un elemento clave en el desarrollo tecnológico y
				económico de nuestra sociedad. Nos invita a conocer la manera en que 
				los datos son recabados y utilizados para la solución de una amplia 
				cantidad de problemas e incluso menciona que el estudio de los mismos 
				se está convirtiendo día a día en una disciplina indispensable para 
				todos los sectores ecónomicos. Concuerdo totalmente con esta postura, 
				pues como bien dice el autor, cantidades enormes de datos se están 
				produciendo diariamente; estos datos contienen información vital para 
				conocer mejor el panorama/las dinámicas que ocurren en algún contexto 
				(sector financiero, agricultura, medicina, producción, etc...) Por lo 
				que aquella empresa incapaz de tomar provecho de estas nuevas tecnologias, 
				simplemente estaría perdiendo la información que necesita para mantenerse 
				vigente.

				El articulo se relaciona con la materia porque su tema principal es 
				justamente el objeto de estudio de esta, ya que sobre nosotros 
				cae la responsabilidad del impacto que podrían tener los datos en los 
				ámbitos sociales, económicos, etc, si no sabemos cómo organizarlos, es decir,
				nosotros tenemos que aprender a analizarlos y a manejarlos para así poder 
				diseñar una base de datos eficiente en donde podamos guardarlos de manera que 
				estos estén seguros, sea fácil poder acceder a ellos, podamos almacenarlos por 
				mucho tiempo, podamos actualizarlos y la forma en la se guarden este organizada 
				y optimizada, y así poder brindarle a las personas una buena herramienta con la 
				que puedan trabajar y que esta sea fácil de usar.
				
				Estoy de acuerdo con la postura del autor, principalmente en la parte de que 
				se pueden utilizar los datos de una manera amigable con el medio ambiente, 
				como bien pueden ser los edificios de energía positiva o de los sensores 
				telemáticos para la captura de datos del rendimiento del motor, con lo que se 
				ahorran millones de galones de combustible, por estas y otras razones se están 
				ahorrando enormes cantidades de energía, y si se hace un mejor uso de los datos 
				y se impulsa más, se pueden llegar a mejorar el medio ambiente a la vez que las 
				empresas ahorran. Otro punto a considerar es que la economía es impulsada por los 
				datos, ya que con un mejor uso de los datos se pueden innovar ciertas áreas 
				económicas e impulsar el empleo, y de hecho se hizo una estimación de que el 
				PIB mundial tendía a crecer por el uso de los datos.
				Pero estos no son los únicos sectores afectados de manera positiva por los datos, 
				tamién se encuentran la medicina, la agricultura y ganadería, las TI, entre otros.
				Creo que el buen uso de los datos nos puede llevar a ser una sociedad más conciente
				con el medio ambiente y puede impulsarnos como sociedad para crecer tanto económicamente 
				como en calidad de vida.\\



			\item Mitos sobre los datos (páginas 17 a 27). Cada integrante del equipo deberá seleccionar al menos 2
				mitos que le hayan parecido curiosos y/o interesantes sobre los datos e indicar qué pensaban antes de
				leerlos y cómo cambio su perspectiva después de la lectura.

				\textbf{Angel}\\
				\textit{Las personas no tienen control alguno sobre sus datos.}\\

				Siento que no se desmiente correctamente este
				mito, es verdad que algunas empresas te permitan
				no ser rastreado muchas no lo hacen y este es
				un caso donde un solo actor malo
				puede tener impactos enormes sobre la vida de
				las personas. En particular tracking
				a través del internet: \href{https://coveryourtracks.eff.org/}{-> un experimento acerca de rastreo digital <-}

				Voy a admitir que si hay un control que todos
				tenemos sobre nuestros datos y es en decidir no
				crearlos, lo que no digitalizas no existe
				en ninguna base de datos.

				\textit{La única forma de que los datos puedan estar protegidos es que
				los gobiernos intervengan para exigir su protección.}\\

				Mi opinión acerca de este mito no cambia,
				fuera de tratarse de detener el progreso de la tecnología
				aislar los datos puede ser una medida para proteger los
				derechos de los ciudadanos de los países.
				Yo no puedo votar acerca de la legislación de datos
				en Estados Unidos pero si en mi propio país, por que
				quería dejar mi información en manos de quienes no
				están obligados a hacerme caso ?
				
				\textbf{Rodrigo}\\
				\textit {Los datos se utilizarán como una herramienta de exclusión a 
				partir de la capacidad de reforzar los obstáculos que enfrentan 
				las comunidades desfavorecidas y de bajos ingresos.} \\
				
				Si bien no creo que los datos son usados para mantener opresión sobre 
				comunidades desfavorecidas, me parece que éstos raramente son usados para
				ayudar a dichas comunidades. Desde mi punto de vista, muchos de los beneficios 
				que las ciencias de datos brindan únicamente son accesibles para personas con un 
				cierto privilegio socioeconómico. Según el artículo, los datos usados de manera 
				\textit{responsable} son una herramienta para estudiar las causas y las soluciones 
				de problemáticas sociales como la marginación, la pobreza, la falta de acceso a la educación,
				entre otras. Desgraciadamente, los datos siguen sin ser usados de la manera adecuada; 
				y en muchos países (incluyendo el nuestro) todas estas situaciones persisten. 
				
				\textit {No se puede confiar en las empresas que utilizan los datos.}\\
        			Con controversias recientes como fue el caso de Facebook, acusado de vender 
				datos personales de sus usarios a mas de 150 empresas, muchas personas (incluyendome) 
				tenemos la sensación de que nuestros datos personales son comerciados deliberadamenete 
				entre empresas con el objetivo de crear publicidad y promover la cultura del consumismo. 
				El artículo nos dice que esto no sucede y que de hecho a las empresas les interesa cuidar 
				la privacidad de sus usuarios. Personalmente no creo que este sea el caso de la mayor 
				parte de las grandes vendedoras de servicios, pero es bueno saber que algunas empresas 
				como Microsoft están adoptando políticas más estrictas para proteger la privacidad de 
				sus clientes.
				
		\end{enumerate}

\end{enumerate}

\end{document}
