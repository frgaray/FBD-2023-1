\documentclass[11pt,letterpaper]{article}
\usepackage[margin=2cm,includefoot]{geometry}
\usepackage[spanish]{babel}
\usepackage{amsmath}

\usepackage{graphicx}
\usepackage{hyperref}
\usepackage{amssymb}
\usepackage{float}

\title{Bases de datos 2023-1\\
Tarea 1: Conceptos básicos}
\begin{document}
\maketitle

\begin{enumerate}
	\item \textbf{Conceptos generales}
		\begin{enumerate}
			\item Describe las principales características del enfoque de bases de datos y contrástalo con el enfoque
				basado en archivos. ¿En qué casos no tendría sentido utilizar una base de datos?

			\item ¿Qué ventajas y desventajas encuentras al trabajar con una base de datos?

			\item ¿Qué es la independencia de datos? ¿Cuál tipo de independencia de datos es más difícil de lograr?
				Justifica tu respuesta.

			\item Explica la diferencia entre los esquemas externo, interno y conceptual. ¿Cómo se relacionan estas
				diferentes capas de esquemas con los conceptos de independencia de datos lógica y física?

			\item Investiga qué es la redundancia de datos. ¿Cuál sería la diferencia entre redundancia de datos
				controlada y no controlada?

			\item Investiga cuáles son las responsabilidades de un DBA. Si asumimos que el DBA nunca está interesado
				en ejecutar sus propias consultas, ¿necesita entender y/o conocer el modelo de datos lógico de la
				base de datos? Justifica tu respuesta.

			\item Entrevista a algún usuario de sistemas de bases de datos, ¿qué características de SMBD encuentran
				más útiles y por qué? ¿qué instalación(es) de SMBD encuentran más/menos complicada y por qué?
				¿cuáles perciben estos usuarios que son las ventajas y desventajas de un SMBD?

			\item Supón que deseas crear un sitio de videos similar a TikTok. Considera cada uno de los puntos
				enumerados en el documento “Purpose of Database Systems”, como desventajas de administrar los
				datos en un sistema de procesamiento de archivos. Discute la relevancia de cada uno de los puntos
				indicados con respecto al almacenamiento de datos de los videos: el usuario que lo subió, la fecha de
				carga, las etiquetas, qué usuarios comentaron, cantidad de “Me gusta”, entre otros.

			\item Investiga por qué surgieron los sistemas NoSQL en la década de 2000 y compara a través de una tabla
				sus características vs. los sistemas de bases de datos tradicionales.

			\item Asumiendo que una base de datos es un lugar donde se almacenan datos de forma sistemática y que
				la información se obtiene al consultar los datos entonces, un diccionario puede considerarse como una
				base de datos. Imagina que vas a buscar el significado de la palabra Luminiscencia, indica cómo
				efectuarías la búsqueda y los problemas que enfrentarías con: un diccionario con palabras
				desordenadas, un diccionario con palabras ordenadas (sin índice) y un diccionario con palabras
				ordenadas (con índice).
		\end{enumerate}

	\item \textbf{Lectura}
		\begin{enumerate}
			\item Leer el resumen ejecutivo ¿Por qué son tan importantes los datos? y realizar un resumen del documento
				(páginas 1 a 16), destacando los puntos que a su consideración sean los más relevantes (no más de
				una cuartilla).

			\item Realizar un ensayo donde expresen sus comentarios (cada integrante del equipo deberá indicar este
				punto de forma individual en el documento que redacten) sobre la lectura, considerando los siguientes puntos :...

			\item Mitos sobre los datos (páginas 17 a 27). Cada integrante del equipo deberá seleccionar al menos 2
				mitos que le hayan parecido curiosos y/o interesantes sobre los datos e indicar qué pensaban antes de
				leerlos y cómo cambio su perspectiva después de la lectura.
		\end{enumerate}

\end{enumerate}

\end{document}
