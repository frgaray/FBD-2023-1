\documentclass[12pt,a4paper]{article}
\usepackage[utf8]{inputenc}
\usepackage[spanish]{babel}
\usepackage[margin=0.5in, top=0.5in, bottom=0.5in]{geometry}
\usepackage{amsmath}
\usepackage{amsfonts}
\usepackage{amssymb}
\usepackage{hyperref}
\usepackage[shortlabels]{enumitem}
\newcommand{\p}{\phantom{......}}

\usepackage{setspace}
\onehalfspacing

\title{Bases de datos 2023-1\\
Tarea 3: Modelo Relacional}
\begin{document}
\maketitle

\begin{enumerate}
	\item Preguntas de repaso.
		\begin{enumerate}
			\item[\textbf{i.}] ¿Qué es una relación y qué características tiene?\\

				Es un subconjunto del producto cartesiano de otros conjuntos (dominios).
				Cada elemento del subconjunto representa una entidad y cada entrada de los
				elementos es una pieza de información de la entidad.

				Las relaciones tienen:

				\textbf{Nombre}: Las identifica de otras relaciones.\\
				\textbf{tuplas}: Elementos de la relación, cada una representa una entidad.\\
				\textbf{atributos}: Elementos de las tuplas que describen propiedades de sus entidades.\\
				\textbf{Dominios}: Los conjuntos a los que pertenecen los atributos, en la entrada $x$
				de una tupla siempre hay elementos del mismo dominio.\\
				\textbf{Cardinalidad}: La cantidad de tuplas en una relación.\\
				\textbf{Grado}: La cantidad de entradas en la tupla (cantidad de atributos).\\
				\textbf{Restricciones}: Estructuras que no se permiten en una relación.\\	


			\item[\textbf{ii.}] ¿Qué restricciones impone una llave primaria y
				una llave foránea al modelo de datos relacional?\\

				\textbf{Primaria}: Unicidad. No pueden existir dos tuplas con los mismo valores
				en los atributos que componen sus llaves primarias.\\

				\textbf{Foránea}: Identificación y existencia. Debe de existir una entidad
				en otra relación que tenga como llave primaria esta llave foránea.\\


			\item[\textbf{iii.}] Investiga que cuáles son las Reglas de Codd y explica con tus propias palabras
				cada una de ellas. Indica por qué consideras que son importantes.\\

				\begin{enumerate}
					\item[0.] Todo sistema de manejo de bases de datos relacionales,
						debe manejar sus bases de datos usando solo relaciones.\\

						Toda la información de las bases de datos está en tablas.\\

					\item[1.] Toda la información de una base de datos relacional se
						representa como valores en tablas.\\

						La información de la base de datos se accesa con tablas.\\

					\item[2.] Todos los datos en una base de datos relacional se deben
						de poder acceder usando solo el nombre de la tabla, una llave primaria
						y el nombre de una columna.\\

						Para extraer un dato único de una base de datos solo se necesita saber
						en que tabla está, a que entidad pertenece y de que columna es.
						Esta regla también impide los valores multivaluados.\\

					\item[3.] Los valores nulos deben ser soportados y representan información
						faltante o inexistente independientemente de un tipo de dato.\\

						Todos los lugares donde puede haber datos también pueden tener null,
						para representar que no hay o no puede haber un dato para esa entrada
						de la tabla.\\

					\item[4.] La descripción de la base de datos se representa en el nivel
						lógico de la misma forma que datos normales para que pueda ser
						manipulada con el mismo lenguaje relacional que ocupan los datos normales.\\

						Debe de haber un catálogo y este debe de poderse modificar con
						el mismo lenguaje que el resto de la base. Esto hace cómodo
						manipular muchas bases de datos al mismo tiempo (entre otras cosas).\\

					\item[5.] Un sistema relacional puede soportar más lenguajes e interfaces, pero
						debe haber almenos uno con expresiones bien definidas en texto y que pueda:
						\{definir datos, ver definiciones, manipular datos, restringir integridad,
						controlar autorización, definir transacciones\}.\\

						Básicamente dice que además de todos los extras que tenga un sistema
						manejador relacional, este debe de poder aceptar ``comandos''
						para todas sus operaciones.\\

					\item[6.] Todas las vistas que se pueden actualizar en teoría pueden
						ser actualizadas por el sistema.\\

						El sistema puede actualizar las tablas de las que dependen
						las vistas si estas en el modelo lógico se pueden actualizar
						sin problemas.\\

					\item[7.] La capacidad de manejar una relación como operando aplica para
						la inserción, actualización y eliminación de los datos.\\

						Insertar, actualizar y eliminar datos de la base se debe poder
						hacer para colecciones de entidades de una relación al mismo tiempo.\\

					\item[8.] Aplicaciones y servicios no son afectados lógicamente cuando
						se cambia el almacenamiento o la forma de acceder a este.\\

						La base de datos no debe ser afectada cuando se mueven los datos,
						se cambia su mecanismo de almacenamiento o se distribuyen.\\

					\item[9.] Aplicaciones y servicios no son afectadas cuando cambios
						que no alteran la información se hacen en las tablas.\\

						La interfase entre las aplicaciones y el sistema manejador
						debe de ser independiente de la estructura de las tablas.\\

					\item[10.] Las restricciones de integridad se deben de poder definir en
						el lenguaje de la base de datos y almacenar en el catálogo, no
						en las aplicaciones o servicios que usen la base.\\

						Las restricciones son resguardadas por el sistema de base de datos,
						por ejemplo, la unicidad de las llaves primarias nos la debe
						de asegurar el sistema, no las aplicaciones que hagan interfaces
						con él.\\

					\item[11.] Un sistema manejador de bases de datos relacionales, tiene
						independencia a la distribución.\\

						Estos sistemas deben de comportarse de la misma manera si importar
						como esta distribuida su información en una red (una sola instancia, muchas).\\

					\item[12.] Si un sistema manejador tiene un lenguaje de bajo nivel,
						este no debe de poder ignorar las restricciones impuestas en
						otro lenguaje de alto nivel.\\

						La intención es que las restricciones puedan ser aseguradas
						en el sistema, que de verdad sean invariantes sin importar
						como se accesa el sistema.\\
				\end{enumerate}

				Son importantes por que limitan lo que puede ser considerado como base de datos relacional.
				Homogenizan los SMDBR para simplificar como son usados por los programas o usuarios.
				También aseguran las necesidades básicas de estos sistemas, lo que es necesario
				para poder utilizarlos eficazmente.\\

				(fuente: \url{https://reldb.org/c/index.php/twelve-rules/} recuperado en 2022-10-03).\\
		\end{enumerate}

	\item Modelo Relacional.
		\begin{enumerate}
			\item[a.] Traduce el siguiente modelo Entidad – Relación a su correspondiente Modelo Relacional.
		\end{enumerate}

	\item Modelo Relacional e inserción de tuplas.

		Considera el siguiente Modelo E/R:

		\begin{enumerate}
			\item[a.] Completa la tabla que se presenta a continuación, convirtiendo el Modelo E-R
				en un Modelo Relacional, para todas las opciones de cardinalidad
				(considera en todos los casos, participación parcial).
				Indica las relaciones resultantes, su llave primaria y la integridad referencial.

				\begin{tabular}{|l|l|}
					\hline
					Modelo E-R	&	Modelo Relacional\\
					\hline
					M:N			&	A(\underline{a1}, \underline{a2}, a3)
					            		&&  B(\underline{b}, b1)
			       			                &&  ab(a1, a2, b, ab1)\\
					\hline
					1:N			&	A(\underline{a1}, \underline{a2}, a3)
							        &&  B(\underline{b}, a1, a2, ab1, b1)\\
					\hline
					N:1			&	A(\underline{a1}, \underline{a2}, b, ab1, a3)
						       	   	&&  B(\underline{b}, b1)\\
					\hline
					1:1			&	A(\underline{a1}, \underline{a2}, a3)
							        &&  B(\underline{b}, b1)
							        &&  ab(a1, a2, b, ab1)\\
					\hline
				\end{tabular}

			\item[b.] Del inciso a) toma el MR que obtuviste para la cardinalidad M : N.
				Asume que los atributos a1, b y ab1 son de tipo entero, mientras que a2, a3 y b1 son de tipo cadena.
				Supón que la relación A tiene 4 tuplas con los siguientes valores
				(2,’ww’,’a’), (4,’xx’,’b’), (6,’yy’,’c’), (8,’zz’,’d’) y la relación B tiene 5
				tuplas identificadaspor los valores 17, 27, 37, 47, 57.
				Los incisos que se presentan a continuación, representan un conjunto de tuplas a
				insertar (en ese orden) en la relación AB, indica cuál conjunto se puede insertar
				completamente en dicha relación. Justifica tu respuesta en cada caso.

				\begin{enumerate}
					\item[i.] (8,’zz’,17,5); (6,’yy’,57,10); (4,’xx’,27,15); (2,’ww’,37,20); (4,’xx’,27,15)\\
					\item[ii.] (17,’zz’,2,’m’); (27,’yy’,4,’n’); (37,’xx’,6,’o’); (47,’ww’,8,’p’); (57,’zz’,4,’q’)\\
					\item[iii.] (2,’a’,17,23); (4,’b’,27,24); (6,’c’,37,25); (8,’d’,47,26); (2,’a’,57,27)\\
					\item[iv.] (2,’ww’,57,’a’); (4,’xx’,37,’a’); (6,’yy’,17,’a’); (8,’zz’,17,’a’); (10,’xx’,27,’a’)\\
				\end{enumerate}
				
				\textbf{Respuesta}\\
				Los conjuntos ii. y iv. quedan descartados ya que los atributos de ab son a1, a2, b, y ab1 (entero, cadena, entero y entero respectivamente). Es decir, tenemos tres atributos de tipo entero y uno de tipo cadena, y las tuplas indicadas en estos incisos no cumplen con estas características de tener 3 entradas enteras y una de cadena.\\
				Respecto a los otros dos conjuntos, podríamos descartar alguno si suponemos cuál es el atributo que corresponde exactamente a a2.
				Es decir, tomando como ejemplo a la tupla de A, (2,'ww','a'), si suponemos que 'ww' corresponde al atributo a2 (es decir que 'ww' funciona como llave de la tupla), entonces podríamos estar tentados a decir que la respuesta sería el inciso i. Sin embargo observemos que hay una tupla repetida, (4,'xx',27,25), por lo que este inciso no es un conjunto y a nosotros se nos pide seleccionar algún conjunto de i. ii. iii. y iv., por lo que descartaremos este inciso (además de que se nos pide \textbf{insertar} completamente el conjunto, que si lo viéramos como conjunto tendríamos que decir que son cuatro tuplas en total, y en este caso sí lo podríamos insertar totalmente, pero si lo que queremos es insertar dos tuplas iguales, esto no se permitiría porque serían datos redundantes).\\
				Por lo que tendremos que suponer que, dentro del mismo ejemplo con la tupla de A, 'a' es el valor que corresponde al atributo a2 de la tupla. Entonces la respuesta sería el inciso iii. ya que para insertar tuplas en la relación ab necesitamos la llave primaria de la tupla de A y de la tupla de B, la de B la tenemos en la tercera entrada de las tuplas (es decir, 17, 27, 37, 47, 57) y las llaves de A también (2 y 'a', 4 y 'b', 6 y 'c', 8 y 'd').

			\item[c.] Del inciso a) toma como base el MR que obtuviste para la cardinalidad 1 : N.
				Los incisos que se presentan a continuación representan un conjunto de tuplas a insertar
				(en ese orden) en la relación B, indica cuál conjunto se puede insertar completamente
				en dicha relación. Justifica tu respuesta en cada caso.

				\begin{enumerate}
					\item[i.] (2,’f’,57,’zz’); (4,’g’,47,’yy’); (6,’h’,37,’xx’); (8,’i’,27,’ww’); (2,’j’,17,’yy’)\\
					\item[ii.] (17,’ww’); (27,’xx’); (37,’yy’); (47,’zz’); (57,’zz’); (17,’xx’); (27,’yy’)\\
					\item[iii.] (57,’f’,8,’zz’); (47,’g’,6,’yy’); (37,’h’,4,’xx’); (27,’i’,2,’ww’); (17,’j’,6,’yy’)\\
					\item[iv.] (57,’f’,8,’a’); (47,’g’,6,’b’); (37,’h’,4,’c’); (27,’i’,2,’d’); (17,’j’,6,’c’)\\
				\end{enumerate}

			\item[d.] Considera el mismo escenario del inciso b para las relaciones A y B.
				Toma como base el Modelo Relacional que obtuviste para la cardinalidad 1:1.
				Supón que tu modelo tiene participación total del lado de la relación A.
				Propón un conjunto de 4 tuplas que se pueda insertar en A y un conjunto que no se pueda insertar
				(también de 4 tuplas). Justifica tu respuesta en cada caso.
		\end{enumerate}

	\item Modelo Relacional y restricciones de integridad.\\

		A continuación, se encuentra el Modelo Relacional de un departamento de recursos humanos de alguna empresa.
		En este esquema, supón que desde es inclusivo, mientras que hasta es exclusivo,
		definiendo el período [desde,hasta).
		Indica cuáles de las siguientes afirmaciones se cumplen y por qué razón
		(sin considerar restricciones adicionales):

		\begin{enumerate}
			\item[a.] Dos departamentos con el nombre ‘Sistemas’ podrían existir al mismo tiempo.\\
			\item[b.] Dos o más empleados pueden administrar el mismo Departamento al mismo tiempo.\\
			\item[c.] Un empleado puede trabajar en un Departamento y administrar otro al mismo tiempo.\\
			\item[d.] Para administrar un Departamento un empleado debe trabajar en dicho departamento.\\
			\item[e.] Un empleado podría trabajar en dos Departamentos a partir de la misma fecha.\\
			\item[f.] Para las tuplas de la relación Administrar, hasta no puede ser anterior a desde.\\
			\item[g.] Dado un empleado, podemos identificar exactamente el Departamento donde trabaja.\\
			\item[h.] Ningún empleado puede cobrar más de un Salario al mismo tiempo.\\
			\item[i.] Algunas tuplas en Salario podrían no tener valor para el atributo desde
				y ningún empleado asociado a ellas.\\
			\item[j.] Un Departamento siempre tiene algún empleado que lo administre\\
		\end{enumerate}

\end{enumerate}
\end{document}
